\documentclass[10pt, a4paper]{article}
% especifico márgenes manualmente
\usepackage[paper=a4paper, left=1.5cm, right=1.5cm, bottom=1.5cm, top=3.5cm]{geometry}
% codificación ISO-8859-1
\usepackage[latin1]{inputenc}
% separación silábica en castellano
\usepackage[spanish]{babel}
% paquetes y caratula de algo2
\usepackage{aed2-symb,aed2-itef,aed2-tad,caratula}

% COMANDOS ESPECIALES
\newcommand{\renombre}[2]{\textbf{TAD} {#1} \textbf{ES} {#2}}

\begin{document}

% CARATULA
\materia{Algoritmos y Estructuras de Datos II}
\submateria{Segundo Cuatrimestre de 2016}
\fecha{\today}
\titulo{Trabajo Pr\'actico 1}
\subtitulo{Especificac\'ion}

\integrante{Barylko, Roni Ariel}{Nro/YY}{mail@dc.uba.ar}
\integrante{Giudice, Carlos}{Nro/YY}{mail@dc.uba.ar}
\integrante{Szperling, Sebasti\'an Ariel}{763/15}{sszperling@dc.uba.ar}
\integrante{Tarr\'io, Ignacio}{Nro/YY}{mail@dc.uba.ar}

\maketitle

% compilar 2 veces para actualizar las referencias
\tableofcontents

\pagebreak
%\newpage

\section{TADs Auxiliares}
	\renombre{posicion}{tupla(nat, nat)}
	
	\renombre{pokemon}{String}

\section{TAD Jugador}
	\begin{tad}{\tadNombre{Jugador}}
		\tadGeneros{jugador}

		\tadExporta{jugador, generadores, observadores b\'asicos}

		\tadIgualdadObservacional{j1}{j2}{jugador}{}

		\tadGeneradores

		\tadObservadores

		\tadOtrasOperaciones

		\tadAxiomas

	\end{tad}

\section{TAD Juego}


\end{document}